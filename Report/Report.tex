% Created 2020-09-25 Fri 09:22
% Intended LaTeX compiler: xelatex
\documentclass[11pt]{article}
\usepackage{graphicx}
\usepackage{grffile}
\usepackage{longtable}
\usepackage{wrapfig}
\usepackage{rotating}
\usepackage[normalem]{ulem}
\usepackage{amsmath}
\usepackage{textcomp}
\usepackage{amssymb}
\usepackage{capt-of}
\usepackage{hyperref}
\setcounter{secnumdepth}{2}
\author{Abhaas Goyal (u7145384)}
\date{\today}
\title{Assignment 1 COMP2310}
\hypersetup{
 pdfauthor={Abhaas Goyal (u7145384)},
 pdftitle={Assignment 1 COMP2310},
 pdfkeywords={},
 pdfsubject={},
 pdfcreator={Emacs 26.3 (Org mode 9.1.9)}, 
 pdflang={English}}
\begin{document}

\maketitle
\setlength\parindent{0pt}


\section{Swarm Behaviour}
\label{sec:orga89fba4}
\begin{itemize}
\item The swarm seems to go to right direction, defined as to go in the direction vector of \textasciitilde{}(1.0, 0.0, 0.0) continuously, in between replenishing itself
\item Also since initially the entities of the swarm are spread apart and reach the globes in different times, their charges also deplete in different times
\item A destination will not become unreachable
\item The center of mass remains the same in this case
\item When done with multiple globes since going in a rightward vector, I want to kil him SMFH
\item If the vehicle is going somewhere and at that point it needs to change the destination problem (limited to one globe)
\end{itemize}
\end{document}